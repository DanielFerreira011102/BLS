\chapter{Background}
\label{c2}

This chapter presents some basic tips and a few examples on how to use \LaTeX.


\section{Language Models}
\label{c2:s:language-models}

\section{Linguistic Complexity}
\label{c2:s:linguistic-complexity}

\section{Biomedical Language Characteristics}

\subsection{Terminology Complexity}
The complexity of biomedical terminology represents a unique challenge for language complexity measurement. 
Medical terms often combine elements from multiple languages, primarily Greek and Latin, creating systematic patterns that affect both structural and cognitive complexity. 
These patterns follow predictable rules but can create significant processing challenges for non-expert readers.

The relationship between professional and lay terminology adds another layer of complexity. 
Many medical concepts can be expressed through either technical or lay terms (e.g., ``myocardial infarction'' vs. ``heart attack''), creating parallel vocabularies that must be considered in complexity measurement. 
The choice between technical and lay terminology significantly affects both comprehension and processing efficiency.