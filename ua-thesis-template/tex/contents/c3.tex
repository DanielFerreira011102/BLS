\chapter{How to reference}
\label{c3}

This chapter presents how to reference (1) documents such as books, journal articles, conference articles, among other works and (2) elements of this document, such as, chapters, sections, subsections, equations, figures, and tables.


\section{Citing other works}
\label{c3:s:citing-other-works}

In this template, it is used the \verb+biblatex+ package for citing other works.
It is important to note that Bib\TeX\ and Bib\LaTeX\ have many similarities but are two different formats.
For example, if you want to indicate a conference's place you have to use the \verb+address+ field in Bib\TeX, but the \verb+venue+ field in Bib\LaTeX.
The \verb+address+ field in Bib\LaTeX\ is used to indicate the publisher's local. If you would like to indicate the publisher in journal articles I suggest you to use the \verb+note+ field.
By default, the references are expected to be in the \verb+refs.bib+ file.
Organize your citations with a reference manager such as, for example, Zotero\footnote{\url{https://www.zotero.org/}} with the Better Bib\TeX\ extension\footnote{\url{https://retorque.re/zotero-better-bibtex/}}.
Zotero is what I personally use and recommend.
For my citation key labels I use the format \texttt{[author][year][a]}: lowercase string with the last name(s) of the first author followed by the year of the publication and a suffix letter for distinction of repeated works (check the \texttt{refs.bib} file for some examples).


\subsection{The citation style}
\label{c3:ss:the-citation-style}

There are several citation styles.
You can configure your own, or use one of the three configurations that are already provided in this template:

\begin{itemize}
\item
Author-year style:\\\verb+% BibLaTeX author-year style.
\usepackage[%
backend=biber,%
% Print back references in the bibliography.
% backref=false,%
backref=true,%
bibstyle=authoryear,%
citestyle=authoryear,%
dashed=false,%
defernumbers=true,%
% To use initials for given (first) names.
% giveninits=true,%
hyperref=true,%
maxbibnames=999,%
minbibnames=1,%
maxcitenames=2,%
mincitenames=1,%
sorting=nyt,%
sortcites=false,%
uniquename=false,%
uniquelist=false,%
]{biblatex}

% To not display first name initials abbreviated.
% https://tex.stackexchange.com/questions/134535/biblatex-authoryear-style-in-text-citations-display-first-name-initials-for-ce

% Adding a comma between author and year.
% https://tex.stackexchange.com/questions/134063/how-to-add-a-comma-between-author-and-year
\renewcommand*{\nameyeardelim}{\addcomma\space}

% Make "et al." italic.
% Make "et al." italic.
% https://github.com/plk/biblatex/issues/899
% https://tex.stackexchange.com/questions/40798/how-do-i-get-et-al-to-appear-in-italics-when-using-textcite-or-citeauthor-w
\DefineBibliographyStrings{english}{
  andothers = {\mkbibemph{et\addabbrvspace al\adddot}}
}
\DefineBibliographyStrings{portuguese}{
  andothers = {\mkbibemph{et\addabbrvspace al\adddot}}
}


% To use square brackets instead of parentheses.
% By default, BibLaTeX uses parentheses.
% % Author-year, square brackets.
% https://tex.stackexchange.com/questions/16765/biblatex-author-year-square-brackets
\makeatletter

\newrobustcmd*{\parentexttrack}[1]{%
  \begingroup
  \blx@blxinit
  \blx@setsfcodes
  \blx@bibopenparen#1\blx@bibcloseparen
  \endgroup}

\AtEveryCite{%
  \let\parentext=\parentexttrack%
  \let\bibopenparen=\bibopenbracket%
  \let\bibcloseparen=\bibclosebracket}

\makeatother


% To also hyperlink the author name.
% By default, BibLaTeX only hyperlinks the year.
% Hyperlink name with biblatex authoryear.
% https://tex.stackexchange.com/questions/1687/hyperlink-name-with-biblatex-authoryear/25972

\DeclareCiteCommand{\cite}
  {\usebibmacro{prenote}}
  {\usebibmacro{citeindex}%
   \printtext[bibhyperref]{\usebibmacro{cite}}}
  {\multicitedelim}
  {\usebibmacro{postnote}}

\DeclareCiteCommand*{\cite}
  {\usebibmacro{prenote}}
  {\usebibmacro{citeindex}%
   \printtext[bibhyperref]{\usebibmacro{citeyear}}}
  {\multicitedelim}
  {\usebibmacro{postnote}}

\DeclareCiteCommand{\parencite}[\mkbibparens]
  {\usebibmacro{prenote}}
  {\usebibmacro{citeindex}%
    \printtext[bibhyperref]{\usebibmacro{cite}}}
  {\multicitedelim}
  {\usebibmacro{postnote}}

\DeclareCiteCommand*{\parencite}[\mkbibparens]
  {\usebibmacro{prenote}}
  {\usebibmacro{citeindex}%
    \printtext[bibhyperref]{\usebibmacro{citeyear}}}
  {\multicitedelim}
  {\usebibmacro{postnote}}

\DeclareCiteCommand{\footcite}[\mkbibfootnote]
  {\usebibmacro{prenote}}
  {\usebibmacro{citeindex}%
  \printtext[bibhyperref]{\usebibmacro{cite}}}
  {\multicitedelim}
  {\usebibmacro{postnote}}

\DeclareCiteCommand{\footcitetext}[\mkbibfootnotetext]
  {\usebibmacro{prenote}}
  {\usebibmacro{citeindex}%
   \printtext[bibhyperref]{\usebibmacro{cite}}}
  {\multicitedelim}
  {\usebibmacro{postnote}}

\DeclareCiteCommand{\textcite}
  {\boolfalse{cbx:parens}}
  {\usebibmacro{citeindex}%
   % \printtext[bibhyperref]{\usebibmacro{textcite}}}
   \printtext[bibhyperref]{\printnames{labelname}\printtext{ \textcolor{black}{\bibopenparen}\printfield{year}\printfield{extradate}\printtext{\textcolor{black}{\bibcloseparen}}}}}
  {\ifbool{cbx:parens}
     {\bibcloseparen\global\boolfalse{cbx:parens}}
     {}%
   \multicitedelim}
  {\usebibmacro{textcite:postnote}}

% The left (first) bracket was being colored.
% I had to slightly modify the above code.
% https://tex.stackexchange.com/questions/1687/hyperlink-name-with-biblatex-authoryear/25972#25972
% https://tex.stackexchange.com/questions/496777/change-color-of-comma-in-citation
% https://tex.stackexchange.com/questions/299050/macro-for-hyperlinked-citation-in-parentheses

+.
\item
Author-year compact style:\\\verb+% BibLaTeX author-year style.
\usepackage[%
backend=biber,%
% Print back references in the bibliography.
% backref=false,%
backref=true,%
bibstyle=authoryear-comp,%
citestyle=authoryear-comp,%
dashed=false,%
defernumbers=true,%
% To use initials for given (first) names.
% giveninits=true,%
hyperref=true,%
maxbibnames=999,%
minbibnames=1,%
maxcitenames=2,%
mincitenames=1,%
sorting=nyt,%
sortcites=false,%
uniquename=false,%
uniquelist=false,%
]{biblatex}

% To not display first name initials abbreviated.
% https://tex.stackexchange.com/questions/134535/biblatex-authoryear-style-in-text-citations-display-first-name-initials-for-ce

% Adding a comma between author and year.
% https://tex.stackexchange.com/questions/134063/how-to-add-a-comma-between-author-and-year
\renewcommand*{\nameyeardelim}{\addcomma\space}

% Make "et al." italic.
% Make "et al." italic.
% https://github.com/plk/biblatex/issues/899
% https://tex.stackexchange.com/questions/40798/how-do-i-get-et-al-to-appear-in-italics-when-using-textcite-or-citeauthor-w
\DefineBibliographyStrings{english}{
  andothers = {\mkbibemph{et\addabbrvspace al\adddot}}
}
\DefineBibliographyStrings{portuguese}{
  andothers = {\mkbibemph{et\addabbrvspace al\adddot}}
}


% To use square brackets instead of parentheses.
% By default, BibLaTeX uses parentheses.
% % Author-year, square brackets.
% https://tex.stackexchange.com/questions/16765/biblatex-author-year-square-brackets
\makeatletter

\newrobustcmd*{\parentexttrack}[1]{%
  \begingroup
  \blx@blxinit
  \blx@setsfcodes
  \blx@bibopenparen#1\blx@bibcloseparen
  \endgroup}

\AtEveryCite{%
  \let\parentext=\parentexttrack%
  \let\bibopenparen=\bibopenbracket%
  \let\bibcloseparen=\bibclosebracket}

\makeatother

+
\item
Numeric style:\\\verb+% BibLaTeX numeric style.
\usepackage[%
backend=biber,%
% Print back references in the bibliography.
% backref=false,%
backref=true,%
bibstyle=numeric,%
citestyle=numeric-comp,%
defernumbers=true,%
hyperref=true,%
maxbibnames=999,%
minbibnames=1,%
maxcitenames=2,%
mincitenames=1,%
sorting=none,%
sortcites=false,%
]{biblatex}
+.
\end{itemize}

You can choose one of these in the \verb+tex/config/packages.tex+ file.
By default, the \verb+authoryear-comp+ style is selected.


\subsection{The citation commands}
\label{c3:ss:the-citation-commands}

A list of citation commands, and examples, follows.

\begin{itemize}
\item
\verb+\cite{oliveiraesilva2012a}+ produces ``\cite{oliveiraesilva2012a}''.
\item
\verb+\cite{oliveiraesilva2012a,oliveiraesilva1994a}+ produces ``\cite{oliveiraesilva2012a,oliveiraesilva1994a}''.
\item
\verb+\textcite{oliveiraesilva2012a}+ produces ``\textcite{oliveiraesilva2012a}''.
\item
\verb+\textcite{oliveiraesilva2012a,oliveiraesilva1994a}+ produces ``\textcite{oliveiraesilva2012a,oliveiraesilva1994a}''.
\item
\verb+\parencite{oliveiraesilva2012a}+ produces ``\parencite{oliveiraesilva2012a}''.
\item
\verb+\parencite{oliveiraesilva2012a,oliveiraesilva1994a}+ produces ``\parencite{oliveiraesilva2012a,oliveiraesilva1994a}''.
\item
\verb+\parencite*{oliveiraesilva2012a}+ produces ``\parencite*{oliveiraesilva2012a}''.
\item
\verb+\parencite*{oliveiraesilva2012a,oliveiraesilva1994a}+ produces ``\parencite*{oliveiraesilva2012a,oliveiraesilva1994a}''.
\item
\verb+\fullcite{oliveiraesilva2012a}+ produces:\\``\fullcite{oliveiraesilva2012a}''.
\end{itemize}

The \verb+\cite{label}+ command is preferred with the numeric style.
In the \texttt{authoryear} and \texttt{authoryear-comp} styles, the \verb+\textcite{label}+ and \verb+parencite{label}+ commands should be used.

Multiple citations are allowed by using multiple labels within the same command.
Example: \verb+\cite{oliveiraesilva2012a,oliveiraesilva1994a}+.

More examples follow.
A \as{phd} thesis~\parencite{oliveiraesilva1994a}.
A \as{msc} dissertation~\parencite{antunes2015a}.
Conference or workshop articles~\parencite{antunes2019a,antunes2020a,almeida2021a}.
A journal article~\parencite{antunes2019b}.


\section{Referencing elements of this document}
\label{c3:s:referencing-elements-of-this-document}

\Cref{c1,c3} are two chapters.

\Cref{c1,c2,c3} are three consecutive chapters.

\Cref{c3:s:citing-other-works,c3:s:a-section} are two sections.

\Cref{eq:example1,eq:example2} are two numbered equations.

\Cref{fig:aveiro} is a figure.

\Cref{tab:a-table-example} is a table.

\Cref{c3:s:a-section,c3:ss:a-subsection} have different levels of depth.

\Cref{a1,a2} are two appendices.


\section{A section}
\label{c3:s:a-section}

This section presents the different levels of depth.


\subsection{A subsection}
\label{c3:ss:a-subsection}

Text in a subsection.


\subsubsection{A subsubsection}

Text in a subsubsection.


\myparagraph{A paragraph}

Text in a paragraph.


\mysubparagraph{A subparagraph}

Text in a subparagraph.
